\documentclass{article}
\usepackage{listings}
\usepackage{xcolor}
\usepackage{CJKutf8}   
\usepackage[UTF8]{ctex}
\usepackage{CJK}
\lstset{
 columns=fixed,
 numberstyle=\tiny\color{gray},                       % 设定行号格式
 frame=none,
 keywordstyle=\color[RGB]{40,40,255},                 % 设定关键字颜色
 numberstyle=\footnotesize\color{darkgray},
 commentstyle=\it\color[RGB]{0,96,96},                % 设置代码注释的格式
 stringstyle=\rmfamily\slshape\color[RGB]{128,0,0},   % 设置字符串格式
 showstringspaces=false,                              % 不显示字符串中的空格
 language=c++,                                        % 设置语言
}
\begin{document}


\begin{lstlisting}
\begin{CJK*}{UTF8}{song}
#include<cstdio>
#include<iostream>
#include<cstring>
#include<string>
#include<algorithm>

using namespace std;

template<class T> int task1(T *a, T b, int n){//模板类函数 0为边界返回
    if (!n) return a[n] == b;
    else return a[n] == b | task1(a, b, n - 1); //位运算取或
}

inline void task2(int n){ //一种位运算枚举子集的方法,可用于枚举任意一个二进制集合的子集
    int tmp = (1 << n) - 1;//因为题目需要全枚举到 所以定义最大集合为 2^n - 1,注意位运算优先级最低
    for (int i = tmp; i; i = (i - 1) & tmp) //不断减1, 并用位运算保证一定是子集
    {
        for (int j = 0; j < n; ++j){//输出方案
            if (i & (1 << j)) printf("1"); // 判断当前位是否为1
            else printf("0");
        }
        printf("\n");
    }
}
int a[10];
int main()
{
    cout<<task1(a, 0, 9)<<endl;
    task2(5);
    system("pause");
}
\end{CJK*}
\end{lstlisting}


\end{document}
